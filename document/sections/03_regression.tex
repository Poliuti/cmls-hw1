\section{Regression} % descrizione di metodi di regressione, selezione feature e cross-validation
Once features have been extracted, regression theory allow to predict a real value from a set of $N$ given inputs, where $N$ is the dimension of the features space. In this context, dimensional \textbf{Music Emotion Recognition} has been considered as a regression problem where two regressors are trained \textit{independently} for valence and arousal. So, the aim is to find two regressors  
\[r_{V, A} \colon \R^{N} \to \R\]
that minimize the mean squared error (\textit{MSE})
\[ \epsilon = \frac{1}{N} \sum_{i =1}^{N} (y_i - r(\vec{x_i}))^2\]
In particular, four regressors will be trained to predict the four-element vector
\begin{center}
[valence mean, valence std, arousal mean, arousal std]
\end{center}
Different regression models are used to find out which set of regressors best fitted data, such as \textbf{Linear Regression}, \textbf{Support Vector Regression} and \textbf{K-Neighbors Regression}. Performance evaluation will be achieved by means of the metrics represented by $R^2$ and $MSE$ statistics. \\
Regressors are implemented as Python module \textit{Scikit-learn} integrating a wide range of state-of-the-art machine learning algorithms \cite{scikit-learn}. Regression training is not computed until the dataset is consistently rearranged and split. In fact, in order to split our collection of features and annotations in the two parts of training and testing set it will be necessary to shuffle our initial dataset so that it will be as inhomogeneous as possible in terms of music genre. Although we will focus on Music Emotion Recognition for the global layer of the song, regression approach is capable to fit also to music emotion variation detection (\textbf{MEVD}) considering the time evolution of features frame-by-frame for each song.\\
As regards the regression models implementation, the best parameters are previously selected through a cross-validation processing accordingly for each regression algorithm. Linear regression is a linear model in which coefficients are computed to minimize the residual sum of squares between the observed targets in the dataset, and the targets predicted by the linear approximation \cite{scikit-learn}. \textbf{Support Vector Regression} is our second attempt and it turned out to be the best approach. \textbf{Support Vector Machines} represents the operation to map our features space to a higher dimensional one and learn a nonlinear function by a linear learning machine in the kernel-induced feature space, where data are more separable \cite{yang2008regression}. The \textbf{SVR} algorithm supplied by \textit{Scikit-learn} provides an easy way to compute different kernels based regressors. Out of some attempts involving linear, polynomial, radial basis function (RBF), and sigmoid kernels ...