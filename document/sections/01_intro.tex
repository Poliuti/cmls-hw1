\section{Introduction}

The aim of this document is to describe the realization of a regressor able to predict the mood of a music piece. In particular, in accordance with the Music Emotion Recognition (MER) theory, we used a dimensional approach to face this problem. We considered two "emotion dimensions" to describe emotions: the \emph{valence}, that describes the grade of pleasantness, and the \emph{arousal}, which represents the level of energy.

The DEAM Dataset was provided to us. We considered 2000 extracts of music pieces, already re-encoded to have the same duration of 45\,s and the same sampling frequency of 44100\,Hz. The annotations related to these songs contain the mean values and the standard deviation for both valence and arousal. The DEAM Dataset also contains, for each piece, a set of values for already extracted features, as the zero crossing rate, the spectrum centroid,..

The first step of our research was the \textbf{feature selection}: we studied the usefulness of the provided data and we extracted other features using the Librosa library in order to have a wider choice. Then we split the dataset to obtain a training and a test set. The next step was to implement the \textbf{regressors}, trying with linear regression and with Support Vector Machine. We also used the method of cross-validation in order to find the best solution. The last step was the \textbf{evaluation} of the prediction, using as metrics the mean squared error and the $R^2$-score.

In the following sections we will analyze in detail our workflow and the tests that led us to our solution of this problem. Further details can be found in the Jupyter Notebook containing the results of our experiment \cite{notebook} and in the GitHub repository of this assignment \cite{github}.

