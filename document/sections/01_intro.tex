\section{Introduction}

The aim of this document is to describe the realization of a regressor able to predict the mood of a music piece. In particular, in accordance with the Music Emotion Recognition (MER) theory, we used a dimensional approach to face this problem. This approach considers emotions on a two dimensional plane: the \emph{valence}, that describes the grade of pleasantness, and the \emph{arousal}, which represents the level of intensity of the emotion.

Our work is based on a provided dataset containing songs, valence-arousal annotations and features extracted from the songs. Section~\ref{sec:dataset} contains a description of the given dataset and of the operations we made on it: dataset splitting, features extraction, features selection, annotations standardization.

The next step was to implement the \textbf{regressors}, trying with different kinds of them and with different parameters, in order to find the best model. The last step was the \textbf{evaluation} of the prediction, using as metrics the mean squared error and the $R^2$-score.

In the following sections we will analyze in detail our workflow and the tests that led us to our solution of this problem. The document is divided in sections, each one of them describing a different topic we considered for our choices. However they are depending on each other, therefore the flow of this paper does not reflect our chronological approach. Further details can be found in the Jupyter Notebook containing the results of our experiment \cite{notebook} and in the GitHub repository of this assignment \cite{github}.

